\documentclass[11pt,a4paper]{article}

\usepackage[spanish]{babel}
\usepackage{amsmath,amsfonts, amssymb, mathtools} % Podemos añadir amssymb, amsthm o bm
\usepackage{graphicx, xparse}
\usepackage[top=2cm,bottom=2cm,left=3cm,right=3cm,marginparwidth=1.75cm]{geometry} % Este paquete permite modificar los márgenes del documento
\usepackage[colorlinks=true, allcolors=blue]{hyperref} % Se indica que los hipervínculos van todos en azul
\usepackage{setspace}
\usepackage{xcolor, tcolorbox}
\usepackage{cancel} %tachar cosas
\tcbuselibrary{breakable}
\usepackage{hyperref}
\usepackage{titlesec}
\usepackage{array}
\usepackage{tikz}
\usetikzlibrary{positioning,shapes.geometric,fit,backgrounds}
\usepackage{listings, lstautogobble}
\usepackage[T1]{fontenc}
\usepackage[utf8]{inputenc}

\graphicspath{ {images/}}


\lstset{
    language=SQL,
    inputencoding=utf8,
    basicstyle=\ttfamily\footnotesize,
    stringstyle=\ttfamily\color{verdeSuave},
    commentstyle=\ttfamily\itshape\color{gray},
    emph={IS, NULLS, ROW, OFFSET, FETCH, WITH, TIES, PERCENT, REFERENCES, TRUNCATE, ADD_MONTHS, LAST_DAY, MONTHS_BETWEEN, NEXT_DAY, ROUND, TRUNC, EXTRACT, TO_CHAR, TO_DATE, TO_NUMBER, MERGE},
    emphstyle={\color{blue}\bfseries},
    keywordstyle=\color{blue}\bfseries,
    morekeywords={SELECT, FROM, WHERE, INSERT, INTO, VALUES, UPDATE, SET, DELETE, JOIN, ON, CREATE, TABLE, DROP, ALTER, ADD, PRIMARY, KEY, FOREIGN, REFERENCES, CONSTRAINT, AND, OR, NOT, NULL, AS, DISTINCT, GROUP, BY, ORDER, HAVING, IN, BETWEEN, LIKE, IS},
    numbers=left,
    numberstyle=\tiny\color{gray},
    xleftmargin=10pt,
    frame=shadowbox,
    breaklines=true,
    extendedchars=true,    % Habilitar caracteres extendidos
    autogobble=true,
    tabsize=4, % Cambia el tamaño del tabulador aquí
    literate=%
        {á}{{\'a}}1 {é}{{\'e}}1 {í}{{\'i}}1 {ó}{{\'o}}1 {ú}{{\'u}}1
        {Á}{{\'A}}1 {É}{{\'E}}1 {Í}{{\'I}}1 {Ó}{{\'O}}1 {Ú}{{\'U}}1
        {ñ}{{\~n}}1 {Ñ}{{\~N}}1
}



%Colores
\definecolor{blanco}{HTML}{FFFFFF}
\definecolor{negro}{HTML}{000000}
\definecolor{azulSuave}{HTML}{6ac9d5}
\definecolor{naranjaSuave}{HTML}{d5956a}
\definecolor{verdeSuave}{HTML}{6ad578}
\definecolor{verdeHoja}{HTML}{006400}
\definecolor{naranjaDuro}{HTML}{bd2d0e}



\titleformat{\part}[display]
  {\normalfont\Huge\bfseries} % Estilo del texto
  {}           % Prefijo antes del título de la parte
  {20pt}                      % Separación entre "Parte I" y el título
  {\Huge}                     % Estilo del título de la parte



%Colores
\definecolor{blanco}{HTML}{FFFFFF}
\definecolor{negro}{HTML}{000000}
\definecolor{azulSuave}{HTML}{6ac9d5}
\definecolor{naranjaSuave}{HTML}{d5956a}
\definecolor{verdeSuave}{HTML}{6ad578}
\definecolor{magenta}{HTML}{FF00FF}
\definecolor{dorado}{HTML}{ad8a1f}
\definecolor{amarilloPastel}{RGB}{255, 249, 196} % Amarillo pastel claro
\definecolor{amarilloOscuro}{RGB}{253, 216, 100}  % Amarillo más oscuro
\definecolor{grisSuave}{RGB}{230, 230, 230}      % Gris suave
\definecolor{negro}{RGB}{0, 0, 0}                % Negro puro

\newtcolorbox{dem_box}[1]{
before=\par\smallskip\centering,
colframe=azulSuave!70,
colback=white,
fonttitle=\bfseries,
coltitle=negro,
title=#1,
flushleft title,
width=1\linewidth,
breakable = true
}

\newtcolorbox{ejem_box}[1]{
before=\par\smallskip\centering,
colframe=verdeSuave!70,
colback=white,
fonttitle=\bfseries,
coltitle=negro,
title=#1,
flushleft title,
width=1\linewidth,
breakable = true
}

\newtcolorbox{ej_box}[1]{
before=\par\smallskip\centering,
colframe=naranjaSuave!70,
colback=white,
fonttitle=\bfseries,
coltitle=negro,
title=#1,
flushleft title,
width=1\linewidth,
breakable = true
}

\newtcolorbox{mod_box}[1]{
before=\par\smallskip\centering,
colframe=amarilloOscuro!85,
colback=amarilloPastel!30,
fonttitle=\bfseries,
coltitle=negro,
title=#1,
flushleft title,
width=0.94\textwidth,
breakable = true
}



\setstretch{1.2}
\decimalpoint

% \title{\textbf{BASES DE DATOS: } EJERCICIOS RESUELTOS}
% \author{Diego Díaz Mendaña}
%\date{Fecha}

\begin{document}

\begin{titlepage}
    \begin{center}
        % Logo de la institución
        \includegraphics[width=0.3\textwidth]{logo_escuela.png} \\[1cm] % Cambiar logo_escuela por el nombre del archivo con el logo
        \vspace{1cm}
        
        \textbf{\LARGE ESCUELA DE INGENIERÍA INFORMÁTICA DE OVIEDO} \\[1.5cm]
        
        \rule{\linewidth}{0.5mm} \\[0.4cm]
        \textbf{\Huge Fundamentos de Computadores Y Redes} \\[0.3cm]
        \rule{\linewidth}{0.5mm} \\[1cm]
        
        {\Large Curso 2024-2025} \\[1.5cm]
        
        \textbf{\LARGE Trabajo Grupal - Fase I} \\[0.8cm]
        
        \text{Díaz Mendaña, Diego - UO301887}\\[0.15cm]
        \text{García Pernas, Pablo - UO??????}\\[0.15cm]
        \text{Gota ?, Jorge - UO???????}\\[0.15cm]
        \text{Suárez Fernández, Fernando - UO300028}
        
        \vfill
        
        % Información del alumno
        \begin{flushright}

            \textbf{Grupo de prácticas:} ? \\[0.3cm]
            \textbf{Titulación:} PCEO Informática y Matemáticas \\[0.3cm]
        \end{flushright}
        
        \vfill
        
        \today \\[1cm]
    \end{center}
  \end{titlepage}

\newpage
\hypersetup{linkcolor=black}
\tableofcontents
\hypersetup{linkcolor=blue}
\newpage

\section{Introducción}
-- Breve explicación

\section{ID}
-- Explicación de la ID

\section{Desarrollo del programa}
\subsection{Descripción general}
-- Descripción general del programa + lenguajes

\subsection{Funciones implementadas}
-- Descripción de las funciones implementadas con: nombre, descripción, lógica, fragmentos relevantes de código, ejemplo de entrada válida, ejemplo de entrada inválida, ejemplo de salida

\section{Cuestiones}

\subsection{Distribución del trabajo}

\end{document}